\section{Introduction}
\label{sec:intro}

As existing \Web{} privacy research has demonstrated, \Web{} trackers use
a variety of techniques to track and violate privacy on the \Web{}. Browser
tracking is usually done through a mix of stateful tracking (\IE{} storing and
transmitting unique identifers in the user's browser) and stateless tracking, or
fingerprinting (\IE{} attempting to uniquely identify a browser based on unique
browser configuration and execution environment characteristics).

Though much recent privacy work has focused on stateless, fingerprinting based
tracking, we expect that the majority of tracking is still done using
traditional stateful methods. This intuition is based on multiple factors,
including…

While the privacy community has had some success in designing defenses
to stateless, fingerprinting tracking that protect users without breaking
benign, user-serving page functionality\cite{laperdrix2017fprandom,
nikiforakis2015privaricator}, the researchers, industry and activists have been
less successful in designing practical, robust defenses against stateful
\TP{} tracking.

Despite the press and attention that the "end of \TP{} cookies" has
received, blocking \TP{} cookies (\IE{} not sending cookies on
requests for \TP{} sub-resources) does not provide any fundamental
protections against stateful \TP{} tracking. Blocking \TP{} cookies is a positive
step for \Web{} privacy, but because it prevents categories of accidental
tracking or information disclosure, not because it prevents intentional
tracking. \TPu{} frames can access the same cookies \footnote{\FNOne{}},
\texttt{localStorage}\footnote{\FNTwo{}}, \texttt{IndexDB}\footnote{\FNThree{}},
or other \JS{} accessible storage methods. In short, blocking \TP{} cookies
is a necessary, but insufficient part of solving the general problem of
preventing stateful \TP{} tracking.

Though some browser vendors have taken some steps to address \TP{} stateful
tracking, each approach has significant shortcomings and limitations.
The details of each technique are described in
Section\ref{sec:background:browsers}, but at a high level some approaches
likely miss significant tracking because they depend on curated lists and
heuristics (\IE{} Firefox and Edge), some approaches address tracking across
sites but not time (\IE{} Safari), and some approaches provide strong
protections against tracking but break sites for users (\IE{} Brave); Chrome
provides no significant defense against \TP{} stateful tracking.

We argue that practical, robust protections against stateful, \TP{} tracking
should have at least three properties.

\begin{enumerate}
  \item \textbf{Cross-site protection}: prevent \TPs{} from using
    stored identifiers to link browsing behavior across \FP{} sites.
  \item \textbf{Longitudinal protection}: prevent \TPs{} from using stored
    identifiers to link browsing behavior on the same \FP{} site across
    time.
  \item \textbf{Web Compatibility}: not effect (or minimally impact)
    user-serving, non-storage behavior in \TP{} frames.
\end{enumerate}

In this work we aim to improve \Web{} privacy by presenting a new method of
managing and limiting \TP{} state that we call "\ToolName{}".
Section \ref{sec:design} presents the approach in detail, but
at a high level, "\ToolName{}" is the unique combination of two features:

First, \textbf{\ToolName{} partitions \TP{} state by the top level document}.
If a browser tab has loaded a page from origin A,
and that page includes two sub-documents (\IE{} \IFrame{}s) from origin B,
the two sub-documents see the same storage, but different storage than B sees
when B is the top-level document, and also different storage than origin B
sub-documents on other pages and tabs.

Second, \textbf{\ToolName{} sets the lifetime of all \TPs{} state to be equal
to the lifetime of the top level document}.  If a page from origin A opens and
closes \IFrames{} from origin B, all of those origin B frames will see the same
storage, even between frames being opened and closed. However, once the page
is closed, all the partitioned storage for B is cleared as well.
Revisiting or reloading the page will result in B seeing empty storage.

More concretely, this work makes the following contributions to improving
privacy on the \Web{}.

\begin{enumerate}
  \item We propose \textbf{a novel approach to managing \TP{} state} in
    \Web{} pages we call \ToolName{}.
  \item We provide an \textbf{open-source, prototype implementation} of
    \ToolName{} as a series of modifications to Chromium.
  \item We define \textbf{metrics for measuring the privacy and
    web-compatibility} properties of an arbitrary \TP{} storage policies.
  \item We release a \textbf{dataset of applying \NumStoragePolicies{} storage
    policies to the \WebDataSet{}}, including the above privacy and
    compatibility metrics.  The \NumStoragePolicies{} policies selected are
    designed to approximate the \TP{} storage policies currently deployed
    in popular browsers.
\end{enumerate}